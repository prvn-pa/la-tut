\documentclass[a4paper,12pt]{article}
\usepackage{amsmath}
\begin{document}
\section{Mathematical Equations}
\subsection{Mathematical Symbols}
Some of the math symbols are \\
\\
$\alpha~~~\beta~~~\gamma~~~\delta~~~\Delta~~~\mu~~~\nu$\\
\\
Arithmetic Operations are \\
\\
$\leq~~~\ll~~~\gg~~~\pm~~~\times~~~\div~~~\neq$\\
\\
Different arrow mark symbols\\
\\
$\rightarrow~~~\Rightarrow~~~\longrightarrow~~~\Longrightarrow$\\
\\
\subsection{Mathematical Equations}
%					example - 00 Using inline math - embed formulas in your text
This formula $f(x) = x^2$ is an example.
\\
%					example - 01 The equation
\begin{equation*}
  1 + 2 = 3 
\end{equation*}

\begin{equation*}
  1 = 3 - 2
\end{equation*}
% 				example - 02 	Align environment
\begin{align*}
  1 + 2 &= 3\\
  1 &= 3 - 2
\end{align*}
%					example - 03 Fractions and more
\begin{align*}
  f(x) &= x^2\\
  g(x) &= \frac{1}{x}\\
  F(x) &= \int^a_b \frac{1}{3}x^3
\end{align*}
% 				example - 04 more sophisticated expressions
\begin{align*}
  f(x) &= x^2\\
  g(x) &= \frac{1}{\sqrt{x}}\\
  F(x) &= \int^a_b \frac{1}{3}x^3
\end{align*}
% 				example - 05	How to write an m x n matrix in LaTeX	
\begin{equation*}
A_{m,n} = 
\begin{pmatrix}
a_{1,1} & a_{1,2} & \cdots & a_{1,n} \\
a_{2,1} & a_{2,2} & \cdots & a_{2,n} \\
\vdots  & \vdots  & \ddots & \vdots  \\
a_{m,1} & a_{m,2} & \cdots & a_{m,n} 
\end{pmatrix}
\end{equation*}
% 			example - 06 
\begin{equation*}
A = 
\begin{pmatrix}
1 & 2 & 3 \\
4 & 5 & 6 \\
7 & 8 & 9
\end{pmatrix}
\end{equation*}
%				example - 07
\begin{equation*}
A = 
\begin{pmatrix}
1 & 2 & 3 \\
4 & 5 & 6 \\
7 & 8 & 9
\end{pmatrix}
\end{equation*}
%				example - 08
\begin{equation*}
B = 
\begin{bmatrix}
a & b & c \\
d & e & f \\
g & h & i
\end{bmatrix}
\end{equation*}
%				example - 09	LateX matrix with no bracket
\begin{equation*}
   \begin{matrix} 
   a_{11} & a_{12} & a_{13}  \\
   a_{21} & a_{22} & a_{23}  \\
   a_{31} & a_{32} & a_{33}  \\
   \end{matrix} 
\end{equation*}
%				example - 10 LateX matrix determinant / vertical bar bracket
\begin{equation*}
   \begin{vmatrix} 
   a_{11} & a_{12} & a_{13}  \\
   a_{21} & a_{22} & a_{23}  \\
   a_{31} & a_{32} & a_{33}  \\
   \end{vmatrix} 
\end{equation*}
%				example - 11 Latex matrix with curly brackets
\begin{equation*}
   \begin{Bmatrix} 
   a_{11} & a_{12} & a_{13}  \\
   a_{21} & a_{22} & a_{23}  \\
   a_{31} & a_{32} & a_{33}  \\
   \end{Bmatrix} 
\end{equation*}
%				example - 12 Latex matrix with double vertical bar brackets
\begin{equation*}
   \begin{Vmatrix} 
   a_{11} & a_{12} & a_{13}  \\
   a_{21} & a_{22} & a_{23}  \\
   a_{31} & a_{32} & a_{33}  \\
   \end{Vmatrix} 
\end{equation*}
%				example - 13 Latex small inline matrix
I love small matrice such $\big(\begin{smallmatrix} a & b\\ c & d \end{smallmatrix}\big)$
%				example - 14 Examples matrix 2 x 2 in LaTeX

\end{document}
