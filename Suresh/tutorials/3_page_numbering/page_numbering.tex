\documentclass[12pt,draft,onecolumn]{article}
\title{How to numbering the pages}
\date{\today}
\author{Suresh Kumarasamy}

\begin{document}
  \maketitle
  \pagenumbering{gobble}
Creating documents with LaTeX is simple and fun. In contrast to Word, you start off with a plain text file (.tex file) which contains LaTeX code and the actual content (i.e. text). LaTeX uses control statements, which define how your content should be formatted. Before you can see what the final result looks like, the LaTeX compiler will take your .tex file and compile it into a .pdf file.

  \newpage
  \pagenumbering{arabic}

Creating documents with LaTeX is simple and fun. In contrast to Word, you start off with a plain text file (.tex file) which contains LaTeX code and the actual content (i.e. text). LaTeX uses control statements, which define how your content should be formatted. Before you can see what the final result looks like, the LaTeX compiler will take your .tex file and compile it into a .pdf file.
\newpage
  \pagenumbering{roman}
Creating documents with LaTeX is simple and fun. In contrast to Word, you start off with a plain text file (.tex file) which contains LaTeX code and the actual content (i.e. text). LaTeX uses control statements, which define how your content should be formatted. Before you can see what the final result looks like, the LaTeX compiler will take your .tex file and compile it into a .pdf file.

\end{document}